\documentclass[a4j, 11pt, dvipdfmx]{jsarticle}

% Required packages
\usepackage{tikz}
\usetikzlibrary{positioning}

\usepackage[
colorlinks=true,
linkcolor=blue,
filecolor=magenta,
urlcolor=magenta]{hyperref}
\usepackage{pxjahyper}

\begin{document}
\title{SlackからGoogle SpreadSheetにデータを転記するシステムの取り扱い説明書}
\author{ \href{https://www.twitter.com/rj\_phys}{Rj.Chiba} \& \href{https://openai.com/blog/chatgpt}{ChatGPT} }
\date{\today}
\maketitle

\section{はじめに}
本取り扱い説明書では、Slackデータ自動転記システムの利用方法について説明します。
本システムは、Slack上のチャンネル上で発言されたデータをGoogleSpreadSheet上に自動で転記することができます。
本書を読み、正しくシステムを利用することで、チームの生産性向上に役立てていただけます。

\section{機能概要}
本システムは以下の機能を有します。

\begin{itemize}
\item Slack APIを使用してチャンネル上の発言を取得
\item Google Apps Scriptを使用して取得した発言をGoogle SpreadSheetに転記
\item 取得したデータは、チャンネルごとにSpreadSheet上に保管される
\item 特定のchannelを保存対象から外すことが可能
\item 一定期間ごとに自動でスクリプトが実行され、Slackのデータが90日で削除されても、データがGoogle SpreadSheet上に保存される
\end{itemize}

\section{手順}
本システムを使用するためには以下の手順を実行してください。

\subsection{Slack Appの設定}
\begin{enumerate}
\item Slack Appの管理者権限を持っているユーザーでログインし、「\href{https://api.slack.com/apps}{Slack API : Application}」を開く。
\item 「Create New App」をクリックする。
\item アプリの名前を入力し、「Create App」をクリックする。
\item 「Add features and functionality」のメニューから「Bots」をクリックする。
\item 「Add a Bot User」をクリックし、Botの名前を入力する。
\item 「OAuth \& Permissions」のメニューから、以下のスコープを追加する。
\begin{itemize}
\item channels:history
\item channels:read
\item groups:history
\item groups:read
\item im:history
\item im:read
\item mpim:history
\item mpim:read
\end{itemize}
\item 「Install App」をクリックし、許可を与える。
\item 「OAuth \& Permissions」の「Bot User OAuth Access Token」をメモする。
\end{enumerate}

\subsection{スプレッドシートの設定}
\begin{enumerate}
\item Google Driveのページを開き、新規スプレッドシートを作成する。
\item 「拡張機能」→「スクリプトエディタ」をクリックする。
\item 作成されたスクリプトエディタで、\href{https://github.com/RjChiba/slackMsgSaver/blob/main/main.js}{Github}にあるコードをコピー\&ペーストする。
\item \verb|const TOKEN = `YOUR_SLACK_API_TOKEN';| に「Bot User OAuth Access Token」を設定する。
\item \verb|const timeZone| にタイムゾーンを設定する(日本国であれば\verb|`Asia/Tokyo'|)。
\item スプレッドシートに戻り、「setups」という名前のシートを作成する。
\item 「setups」の「A1」から「A9」まで、次の項目を入力する。

\begin{table}[htp]
\caption{「setups」に記載する内容}
\begin{center}
\begin{tabular}{|c|c|}
\hline
セル & 内容\\
\hline
A1 & 前回実行日時\\
A2 & 次回実行日時\\
A3 & 実行インターバル(日)\\
A4 & 保存対象外のチャンネル(1)\\
A5 & 保存対象外のチャンネル(2)\\
A6 & 保存対象外のチャンネル(3)\\
A7 & 保存対象外のチャンネル(4)\\
A8 & 保存対象外のチャンネル(5)\\
A9 & 保存対象外のチャンネル(6)\\
\hline
\end{tabular}
\end{center}
\end{table}%

\item 「setups」の「B4」から「B9」まで、保存対象外のチャンネルIDを入力する。
\end{enumerate}

\subsection{システムの立ち上げ}
\begin{enumerate}
\item 再度スクリプトエディタを開く。
\item メニューバーから関数\verb|main|を選択し、「実行」をクリックする。
\item 「このアプリはGoogleで確認されていません」というポップアップが表示されるので、「詳細」→「slackMsgSaverに移動」をクリック。
\item 「許可」をクリックし、プロジェクトに権限を与える。
\end{enumerate}


\begin{figure}[h]
\centering
\begin{tikzpicture}[node distance=2cm]

\node (gas)      [rectangle, draw, minimum width=3cm, minimum height=1.5cm, align=center] {Google Apps Script};
\node (slackapi) [rectangle, draw, minimum width=3cm, minimum height=1.5cm, align=center, below of=gas] {Slack API};
\node (slackws)  [rectangle, draw, minimum width=3cm, minimum height=1.5cm, align=center, below of=slackapi] {Slack Workspace};
\node (gsheet)   [rectangle, draw, minimum width=3cm, minimum height=1.5cm, align=center, right of=gas, xshift=6cm] {Google SpreadSheet};

\draw [ ->] (gas.south) -- node[right] {1. API Call} (slackapi.north);
\draw [<->] (slackapi.south) -- node[right] {2. データ取得} (slackws.north);
\draw [ ->] (slackapi.north) -- node[left] {3. データ送信} (gas.south);
\draw [ ->] (gas.east) -- node[below] {5. データ書き込み} (gsheet.west);

\end{tikzpicture}
\caption{システムの模式図}
\label{fig:system}
\end{figure}


\end{document}